% Gemini theme
% https://github.com/anishathalye/gemini

\documentclass[final]{beamer}

% ====================
% Packages
% ====================

\usepackage[T1]{fontenc}
\usepackage{lmodern}
%\usepackage[size=custom,width=120,height=72,scale=1.0]{beamerposter}
\usepackage[size=a0,scale=1.1]{beamerposter}
\usetheme{gemini}
\usecolortheme{mit}
\usepackage{graphicx}
\usepackage{booktabs}
\usepackage{tikz}
\usepackage{pgfplots}
\usepackage[caption=false,font=small]{subfig}
%\usepackage{subcaption}
% ====================
% Lengths
% ====================

% If you have N columns, choose \sepwidth and \colwidth such that
% (N+1)*\sepwidth + N*\colwidth = \paperwidth
\newlength{\sepwidth}
\newlength{\leftcolwidth}
\newlength{\maincolwidth}
\newlength{\rightcolwidth}
\setlength{\sepwidth}{0.025\paperwidth}
\setlength{\maincolwidth}{0.3\paperwidth}
\setlength{\leftcolwidth}{0.3\paperwidth}
\setlength{\rightcolwidth}{0.3\paperwidth}

\newcommand{\separatorcolumn}{\begin{column}{\sepwidth}\end{column}}

% ====================
% Title
% ====================

\title{Stopping Silent Sneaks: Defending against Malicious Mixes through Topological Engineering}

\author{\textbf{Xinshu Ma} \inst{1} \and Florentin Rochet \inst{2} \and Tariq
 Elahi \inst{1}}

\institute[shortinst]{\inst{1} \Large University of Edinburgh
\samelineand \inst{2} University of Namur}
%\samelineand \inst{2} U.S. Naval
 %Research Lab \samelineand \inst{3} Princeton University}

% ====================
% Body
% ====================

\begin{document}

\begin{frame}[t]
\begin{columns}[t]
\separatorcolumn

\begin{column}{\leftcolwidth}

%  \begin{block}{Goals}
%    Mitigate private infrastructure threats due to adversarial resources (META)
%    
%    \textbf{Reduce the end-to-end compromised traffic in mixnets while optimising tradeoff between security and performance.}
%    
%%    \begin{itemize}
%%      \item \textbf{Focus on the impact of topological construction on mixnets
%%      Investigate topological, no topology right now.
%%      }
%%      \item \textbf{Optimise tradeoff between security and performance
%%      Balance the tradeoffs between security and performance}
%%      \item \textbf{Reduce end-to-end compromise
%%      %Optimal path selection for further improvement on security.
%%      }
%%    \end{itemize}
%
%  \end{block}

\begin{block}{Problem}

\heading{Trustworthy Mixnet Construction from Untrusted Resources}


  \begin{figure}
    \includegraphics[width=\rightcolwidth]{images/problem_overview.pdf}
    \caption{\textbf{Anytrust assumption} might break in the real world: users suffer from \textbf{end-to-end compromise} and \textbf{client enumeration} by passive adversary.}
  \end{figure}
  
  \vspace{-1cm}

    \begin{figure}
    \includegraphics[width=\rightcolwidth]{images/problem_max.pdf}
    \caption{Adversary \textbf{maximizes} the compromised rate by achieving even distribution.}
  \end{figure}
  
  \vspace{1.5cm}
 \heading{Mixnet construction model:}
    \begin{figure}
    \includegraphics[width=\rightcolwidth]{images/problem_model.pdf}
    \caption{3-stage process: periodically reconstruction; only use a subset of nodes.}
  \end{figure}
  
  \vspace{1.5cm}
\heading{Example of Adversary's Manipulation}
  \begin{figure}
    \includegraphics[width=\rightcolwidth]{images/problem_eg.pdf}
    \caption{Adversary can compromise more than $10\%$ of total traffic with $\alpha=0.2$ resources by manipulating mixes' bandwidth.}
  \end{figure}
\end{block}
%  \begin{block}{Code availability}
%    \begin{center}
%      See me on github!\\
%      https://github.com/CLAPS-CCS2020
%      \begin{figure}
%        \centering
%      \includegraphics[scale=0.6]{images/githubicon.png}
%      \end{figure}
%      \vspace{-0.7cm}
%    \end{center}
%  \end{block}
%    \begin{center}
%      \vspace{5cm}
%      \begin{figure}
%        \includegraphics{images/tor_logo.png}
%      \end{figure}
%    \end{center}
\end{column}


\separatorcolumn

\begin{column}{\maincolwidth}

%  \begin{alertblock}{Evaluation Framework}
%
%  \begin{center}
%    \heading{A generic framework to reason on mixnet topological construction}
%  \end{center}
%    
%    \begin{minipage}[t]{0.31\maincolwidth}
%      \begin{center}
%      \heading{Mitigate Adversary's Manipulation} 
%       \begin{itemize}
%         \item Establish a construction algorithm that limits the adversary's advantage
%         \item Quantify the advantage getting from adversarial manipulation
%       \end{itemize}
%     \end{center}
%    \end{minipage}
%    \begin{minipage}[t]{0.31\maincolwidth}
%      \begin{center}
%      \heading{Bound Users' Exposure to Malicious Mixes} 
%       \begin{itemize}
%         \item Establish guard design in mixnet layered routing.
%         \item Quantify users' guard exposure rate
%       \end{itemize}
%     \end{center}
%    \end{minipage}
%    \begin{minipage}[t]{0.31\maincolwidth}
%      \begin{center}
%      \heading{Achieve Sustainable Performance} 
%       \begin{itemize}
%         \item Design sampling and placement strategies.
%         \item Quantify the overhead of message processing and mixnet construction
%       \end{itemize}
%     \end{center}
%    \end{minipage}
%  \end{alertblock}
% \vspace{1cm}

  \begin{alertblock}{Design}
  \heading{Challenges}
  \begin{enumerate}
    \item \textbf{An adversary's manipulation is hard to detect.}
    \item \textbf{An adversary can perform client enumeration at a low cost.}
    \item \textbf{Tolerating nodes churn.}
    \item \textbf{Performance bottleneck comes from the layer with lowest capacity.}
  \end{enumerate}
      
    \heading{Our Approach}
    \begin{enumerate}
    \item \textbf{Contradictory requirements creates challenges for even distribution.}
    \item \textbf{Guard layer mitigates the risk of client enumeration.}
    \item \textbf{Stability tracking and network maintenance design.}
    \item \textbf{Bin-packing placement to reach even layer capacity.}
    \end{enumerate}
    
    \vspace{2cm}
    \heading{Bow-Tie: High-level Overview}
    \begin{figure}[t]
    \includegraphics[width=0.95\rightcolwidth]{images/bowtie_init.pdf}
    \caption{Mixnet Initialization}
    \end{figure}
%     \vspace{1cm}
%    \begin{figure}[t]
%    \includegraphics[width=0.95\rightcolwidth]{images/bowtie_maintain.pdf}
%    \caption{Mixnet Maintenance}
%    \end{figure}
    \vspace{1cm}
    \begin{figure}[t]
    \includegraphics[width=0.95\rightcolwidth]{images/bowtie_routing.pdf}
    \caption{Mixnet Routing}
    \end{figure}
    \end{alertblock}
    
    \vspace{1cm}
    \begin{alertblock}{Takeaways}
    \begin{itemize}
    \item \textbf{Design: A constrained guard layer that is populated with stable and high performant relays creates a challenge for the adversary.}
    \vspace{1cm}
    \item \textbf{Results: Bow-Tie finds a good balance between security and performance.}
    \end{itemize}
  \end{alertblock}

% \begin{alertblock}{Results -- Security, Performance, and Efficiency}
%    \begin{minipage}[t]{0.95\maincolwidth}
%      \vspace{-1cm}
%      \begin{figure}[t!]
%        \centering
%        \subfloat[h=0.35]{\includegraphics[width=0.33\textwidth]{images/best_bw_35.pdf}}
%        \subfloat[h=0.55]{\includegraphics[width=0.33\textwidth]{images/best_bw_55.pdf}}
%        \subfloat[h=0.75]{\includegraphics[width=0.33\textwidth]{images/best_bw_75.pdf}}
%    \caption{Fully compromised rate versus bandwidth per injected malicious node, as a bandwidth budget of 
%    $2280$ MBps is evenly distributed over a varying number of malicious mix nodes, for each mixnet 
%    construction strategy. $h$ denotes the threshold at sampling step.}
%      \end{figure}
%      \vspace{1cm}
%\begin{figure}[t!]
%    \centering
%    \subfloat[No guard]{\includegraphics[width=0.33\textwidth]{images/ttfb_static_simple.pdf}}
%    \subfloat[With guard]{\includegraphics[width=0.33\textwidth]{images/ttfb_static_simple_withguard.pdf}}
%    \subfloat[Individual Risks]{\includegraphics[width=0.33\textwidth]{images/ttfb_email_authors.pdf}}
%	%\vspace{-1em}
%    \caption{Average time to fall into a fully compromised route for the clients. 
%    Figure (a) shows insufficient security if only applying topological construction algorithms. 
%    Figure (b) shows discrepancies when our guard proposal is
%used for different mixnet topological construction algorithms. 
%Figure (c) shows differences when users communicate with different patterns .}
%    \label{fig:placement}
%\end{figure}
%\vspace{1cm}
%\begin{figure}[t!]
%    \centering
%    \subfloat[Bw-weighted path selection]{\includegraphics[width=0.33\textwidth]{images/waittime_bw.pdf}}
%    \subfloat[Random path selection]{\includegraphics[width=0.33\textwidth]{images/waittime_rand.pdf}}
%    \subfloat[Runtime per iteration]{\includegraphics[width=0.33\textwidth]{images/Runtime.pdf}}
%%	\vspace{-1em}
%    \caption{Propagation time and algorithm runtime.
%       \vspace{0em}}
%    \label{fig:placement}
%\end{figure}
%    \end{minipage}
 
%    \hfill
    
%    \begin{minipage}[t]{0.04\maincolwidth}
%    \end{minipage}
%.   \hfill


%.   comment the minipage to remove the descriptions
%    \begin{minipage}[t]{0.23\maincolwidth}
%          \begin{center}
%      \heading{Security Evaluation 1: \\Fully compromised Traffic} 
%       \begin{itemize}
%         \item The fraction of traffic that are routing via fully compromised paths.
%         %\item The fraction of sampled candidate mixes goes up, the adversary observes less traffic (BwRand).
%         \item The adversary achieves much higher fully compromised rate if the mixnet is constructed by BwRand.
%         \item The fraction of sampled mixes goes up, the adversary observes less traffic of the total.
%       \end{itemize}
%     \end{center}
%
%       
%          \begin{center}
%      \heading{Security Evaluation 2: \\Time to first compromise} 
%       \begin{itemize}
%       	 \item Expected time for the users to choose a fully compromised route.
%	\item With guards, users' security is further improved.
%	%\item Bow-Tie carefully tries to minimize client?s exposure to different guards.
%       \end{itemize}
%     \end{center}
%      
%         \begin{center}
%      \heading{Performance Evaluation: \\Average propogation time} 
%       \begin{itemize}
%         %\item The expected latency, no matter what algorithm, is pretty low with bw-weighted path selection.
%         \item The average time it takes in average to process the message in three layers.
%         \item Bw-PS method handles much more throughput within fixed period of time.
%       \end{itemize}
%     \end{center}
%     
%              \begin{center}
%      \heading{Efficiency Evaluation: \\Algorithm runtime} 
%       \begin{itemize}
%         \item The time it takes to construct a mixnet.
%         \item BwRand consumes highest time while Bow-Tie is the fastest algorithm.
%       \end{itemize}
%     \end{center}
%     
%    \end{minipage}
  
%  \end{alertblock}

\end{column}

\separatorcolumn

\begin{column}{\rightcolwidth}

\begin{block}{Analysis}


\heading{A Balance between Security and Performance}
      \begin{figure}[t!]
        \centering
        \subfloat[End-to-end compromise rate]{\includegraphics[width=0.45\textwidth]{images/worst_case_a02_bw.pdf}}
%        \subfloat[h=0.55]{\includegraphics[width=0.33\textwidth]{images/GE_cdf.pdf}}
        \subfloat[Average queuing delay]{\includegraphics[width=0.45\textwidth]{images/waittime_bw.pdf}}
    \caption{Adversary controls $\alpha=0.2$ of the total bandwidth. $h$ denotes the active pool size at the sampling step.}
      \end{figure}


\heading{Necessity of Guard Design}
\begin{figure}[t!]
    \centering
    \subfloat[Bow-Tie vs others]{\includegraphics[width=0.33\textwidth]{images/full_ttfc_cdf_routesimresults.pdf}}
    \subfloat[Turn of guard logic]{\includegraphics[width=0.33\textwidth]{images/no_guard_ttfc_cdf_routesimresults.pdf}}
    \subfloat[Turn on guard logic]{\includegraphics[width=0.33\textwidth]{images/dynamic_200guard_ttfc_cdf_routesimresults.pdf}}
	%\vspace{-1em}
    \caption{The combination of guard layer and client-side guard logic reduces clients' exposure more effectively than they each could do.}
    \label{fig:placement}
\end{figure}


\heading{Influence of Protocol Design and User Behaviour}
\begin{figure}[t!]
    \centering
    \subfloat[Influence of protocol design]{\includegraphics[width=0.45\textwidth]{images/time_loopix_vs_vuvuzela.pdf}}
    \subfloat[Influence of individual behaviours]{\includegraphics[width=0.45\textwidth]{images/ttfb_email_authors.pdf}}
%    \subfloat[Runtime per iteration]{\includegraphics[width=0.33\textwidth]{images/Runtime.pdf}}
%	\vspace{-1em}
    \caption{(a) Bow-Tie's effect is compatible to mixnet protocol designs. (b) Users can figure out how long they could safely use the network based on their behaviours.}
    \label{fig:placement}
\end{figure}
\end{block}

\vspace{-1cm}
  \begin{block}{Contact}
  \begin{minipage}[t]{0.2\rightcolwidth}
  \end{minipage}
  \hfill
  \begin{minipage}[t]{0.3\rightcolwidth}
 \centering
    \textbf{x.ma@ed.ac.uk}\\
    \textbf{frochet@ed.ac.uk}\\
    \textbf{t.elahi@ed.ac.uk}
  \end{minipage}
  \hfil
  \begin{minipage}[t]{0.3\rightcolwidth}
  \vspace{-1cm}
  \begin{figure}
  \includegraphics[width=0.5\textwidth]{images/artifact.png}
  \end{figure}
  \end{minipage}
  \hfill
  \begin{minipage}[t]{0.2\rightcolwidth}
  \end{minipage}
  \end{block}
  
  
  
 \begin{minipage}[t]{0.32\rightcolwidth}
   \begin{figure}
   \centering
    \includegraphics[width=0.8\textwidth]{images/UoE-logo.png}
  \end{figure}
  \end{minipage}
  \hfil
  \begin{minipage}[t]{0.32\rightcolwidth}

  \begin{figure}
    \centering
    \includegraphics[width=0.8\textwidth]{images/unamur-logo.jpeg}
  \end{figure}
  \end{minipage}
  \hfill
  \begin{minipage}[t]{0.33\rightcolwidth}
\vspace{-1cm}
   \begin{figure}
     \centering
      \includegraphics[width=0.8\textwidth]{images/REPHRAIN-logo.pdf}
  \end{figure}
  \end{minipage}


%  \begin{block}{Key Takeaways}
%    There have been a lot of exciting findings in this work, including key takeaways as follows:
%    \begin{itemize}
%      \item Bow-Tie provides higher level security than others presently; it enables good performance and efficiency as well.
%      %\item Bow-Tie balances the tradeoff between security and performance.
%      \item Using guards in layered routing for mixnets lowers the risk of users being de-anonymised.
%      \item Bandwidth-weighted path selection provides better load balance and security level.
%    \end{itemize}
%  \end{block}



 % \begin{figure}
      %\includegraphics[scale=2]{images/logo.png}
  %\end{figure}
  %\begin{figure}
    %\includegraphics[scale=0.7]{images/innoviris-brussels-empowering-research.png}
  %\end{figure}

\end{column}

\separatorcolumn
\end{columns}
\end{frame}

\end{document}
